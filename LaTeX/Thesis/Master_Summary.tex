\chapter{Summary}
Evolutionary biologists are trying to reconstruct genomes of extinct ancestral species.
All genetic information is subject to continuous changes called mutations.
Mutations are not only made up at the level of base-pair changes but also by rearrangement events such as inversions,
translocations, fusions and fissions~\cite{Kent30092003}.
\\ \ \\
In order to model these events, a notation from previous work is used~\cite{Bergeron2006}.
Let $g$ be a conserved region (\emph{gene}) in a genome.
Each gene has two extremities, the tail ($g^t$) and the head ($g^h$).
We write $g$ ($-g$) if along the chromosome the tail of $g$ comes before (after) its head.
Two adjacent extremities of two consecutive genes form an \emph{adjacency}.
Considering linear chromosomes, an extremity without an adjacent extremity is called \emph{telomere}.
One rearrangement model that considers the most common rearrangement events is the \emph{\gls{dcj}} model~\cite{Yancopoulos15082005}.
\\ \ \\
In order to reconstruct ancestral gene orders, internal nodes of a given phylogenetic tree, with extant genomes at the leaves, have to be labeled
with the ancestral genomes in a way such that the cost of the resulting tree is minimized.
This is also known as the \emph{\gls{spp}}. 
Over the past few years, different methods to reconstruct ancestral gene orders have been proposed.
These methods can be divided into two groups: event-based methods and homology-based methods.
Event-based methods use rearrangement models and scenarios to reconstruct the ancestral genome.\\
Transforming one genome into another with the help of rearrangements is also called \emph{sorting}.
Every series of rearrangement operations that transforms genome $A$ into $B$ is a \emph{valid sorting scenario}.
However, it has been shown that the solution space of sorting with \gls{dcj} grows exponentially with the distance of the genomes to be sorted~\cite{Stoye2010}.
Furthermore, solving the \gls{spp} under the \gls{dcj} model is NP-hard~\cite{Tannier2009}.
\\ \ \\
Recently Feijão~\cite{Feijao2015} proposed a new heuristic that solves the \gls{spp} under \gls{dcj}.
For that, he used the \emph{\gls{cbp}} that visualizes the rearrangement events between two genomes $A$ and $B$.
Using the \gls{dcj} model in order to calculate optimal sorting scenarios exploits the advantage of the \gls{cbp}
where every component of the graph is a cycle.
The goal in that work is to find intermediate genomes of two genomes $A$ and $B$ that arise in optimal sorting scenarios.
Labeling the internal nodes of the phylogenetic tree with intermediate genomes improves the ancestral reconstruction results, even though the tree itself
might not be the most parsimonious one~\cite{Feijao2015}.
\\ \ \\
\todo{Dieser Abschnitt muss zum Schluss neu geschrieben werden!}\\
\todo{Kurzen Absatz über die Aufteilung der Masterarbeit}
In the scope of my master thesis I want to extend the method of Feijão by a probabilistic approach.
This approach considers the probability of an adjacency being affected by a rearrangement.
Similar approaches have been proposed for event-based methods already~\cite{Ma2006,Yang2014,Hu2013}, however the framework of Feijão is able to 
add information about homology. 
With this information, we think we are able to improve the probabilistic approaches.\\
Another point I would like to study is the positional constraint of genes.
Swenson and Blanchette~\cite{Swenson2015} showed how to color adjacency graphs considering the position of genes \emph{in vivo}.
These positional constraints could improve the probability approach even further.